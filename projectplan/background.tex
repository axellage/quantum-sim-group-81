\section{Background}
% Bakgrund ska innehålla en motivering till varför det valda ämnet är intressant ur akademisk synvinkel och/eller ur tekniskt perspektiv eller i förekommande fall ur kundens/uppdragsgivarens perspektiv. I vissa fall ska den här rubriken inkludera en kort historik över ämnet. Efter att ha läst bakgrunden ska alla läsare förstå varför ämnet är relevant. Följande frågeställningar bör vara aktuella:

% Vad är ämnet/problemet som ska undersökas? Varför har ämnet/problemet uppkommit? Varför och för vem är det ett relevant eller intressant ämne/problem? Kan det specifika ämnet/problemet relateras till en mer generell diskussion?
\subsection{History} % strukturen borde bli tydligare, att introduktionen här bara är en introduktion till historien och sen handlar subsubsektionerna om olika perioder i utvecklingen
Quantum mechanics is one of the most widely recognised and mentioned branches of modern physics, with greats like Albert Einstein and Erwin Schrödinger pioneering the field while simultaneously being two of the field's most well known critics. % plöstlig vändning i meningen /chiara
The question arose in the early 1980's whether it was possible to achieve faster than light signaling using quantum mechanical effects, which according to Einstein's theory of relativity is impossible\cite{niel_chang}. % whether istället för if?
The answer to this question was discovered to depend on the possibility of cloning an unknown quantum state, i.e. creating a copy of said quantum state. Exploration of this question resulted in the \textit{no cloning} theorem, which in short states that it is in general impossible to clone a quantum state. This theorem is one of the first results of quantum computation\cite{niel_chang}. 

\subsubsection{Proposal of a Quantum Computer}
As the field of quantum mechanics increased in popularity amongst scientists in the 1980's, a major problem arose. As Richard P. Feynman pointed out in his famous lecture \textit{Simulating Physics with Computers}\cite{feyn} at the first conference on the physics of computers in 1981\cite{brief}, it is very difficult to simulate quantum mechanical systems on classical computers. This is due to the fact that the memory needed to simulate a quantum mechanical system grows exponentially, thus quickly requiring more memory than any classical computer could provide\cite{feyn}. In addition to this will a classical computer only achieve an approximation of said quantum system. Feynman understood that in order to exactly simulate quantum mechanical systems the simulator itself needs to be a quantum mechanical system\cite{feyn}. 

\subsection{Possibilities}
Quantum computers are still in an early stage of development and little is known about what they will be capable of, however; the results and theories are promising. To start, quantum computing is offering a new paradigm in the field of computation and science in general. Much like when the first programmable general-purpose computer ENIAC was built in the 1940's, which was the size of a basement and used purely for calculations\cite{eniac}. There was hope that it would speed up the process of tedious calculations, which it did, however there were no thoughts about it leading to portable and pocket sized devices capable of photography, capturing video, streaming movies, playing games and worldwide connection. 

Due to the fact that quantum computers essentially are quantum systems and will therefore excel at simulating quantum systems, they enable major possibilities in studying quantum mechanics and simulating atoms and molecules. This in turn has the possibility of revolutionising materials science and medicine. Scientists will be able to accurately simulate new complex materials and drugs instead of having to produce and test them\cite{harvard}. In addition to this quantum computers are superior to classical computers in factorising large numbers, which makes them capable of breaking the encryption used today in cyber security e.g. storing personal and banking data etc. This is demonstrated in \textit{Shor's algorithm} invented by Peter Shor in 1994\cite{niel_chang}.

\subsection{Simulating a Quantum Computer}
Despite promising results, quantum computers are early in their development with only a few working quantum computers built in the entire world, still too small for being usable in the aforementioned applications. In order to study quantum computer behaviour despite this inaccessibility, one can simulate a quantum computer using a classical computer. Therefore, this thesis is about building such a simulator, where the behavior of the quantum computer is simulated by a classical program which approximates the results of the quantum operations. 



