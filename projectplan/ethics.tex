\section{Ethics}
% I planeringsrapporten förväntas gruppen skriva en kortare text där gruppen bedömer om samhälleliga och etiska aspekter behöver beaktas och analyseras vidare i uppsatsen/rapporten. Gruppen använder sig med fördel av bilaga 7 som stöd samt de digitala resurser som finns på Studentportalens sidor om kandidatarbetet.

As with all technology there are both good but also malicious uses for it.
One such harmful use of quantum computing is the usage of Shor's algorithm to break RSA-encryption making it obsolete once we have sufficiently powerful quantum computers.
This means that in the future it might not be possible to encrypt data over the internet, which would break many vital functions of our society, for example digital monetary transactions and online identification.
However, this is a problem that is currently being worked on \cite{NIST} and as today's quantum computers are from being powerful enough, with the most powerful one only having a thousandth of the necessary qubits \cite{Forbes}, it is not unlikely that a solution will have been found before that point is reached.
In this project we are not developing such a computer but one could argue that we are in the long term indirectly contributing to it by contributing to the quantum computing field.
On the other hand quantum computing potentially has many positive uses, such as diagnosing diseases \cite{quantum-medicine-diagnosis}, developing new drugs \cite{quantum-medicine-drugs} and the development of artificial intelligence \cite{quantum-ai}.
The third of those three is a complicated subject in itself but it might in turn also bring the potential of positive changes.
Because of these possible uses for the technology we believe it is worth it to keep improving it despite the risks.
The ideal scenario would of course be if no one used quantum computing for malicious uses, such as hacking, but since that is impossible we are left with the option of hoping that the benefits of the technology outweighs the drawbacks.