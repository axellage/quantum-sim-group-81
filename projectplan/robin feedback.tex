Thanks! Here are my comments.

Overall very well written! Well done. I think it's clear you've thought about both the content of the project and your process, and you've already understood some difficult concepts in quantum computing.

Page 2

Line 3: "two of the field's most well known critics" - well, sort of. I know that's often said, but really what they wrote is part of the normal debate within science, and is how a theory develops over time. The usual tabloid way of describing things - somebody suggesting a way to modify a theory means they are "against" the theory or the theory is in "crisis" - is a misunderstanding of the scientific method.

But that's my opinion - write what you believe.

Line 3: "To Einstein’s probable dismay, the question arose in the early 1980’s..." - this is an odd sentence since Einstein died in 1955.

Section 1.2 line 10: "it enables major possibilities" -> "they enable major possibilities" or "they will enable major possibilities"

Section 1.3 line 1: "Despite promising results are quantum computers" -> "Despite promising results, quantum computers are"

Good explanation of the purpose

Section 2.1 Line 1: "The projects" -> "The project's"

Section 3.1: Not really necessary to give citations for Newton and Einstein, since they're not relevant to your project.

Line 2: "If we input the same values on one side of the equation, we always get the same result." Well, this is true for the equations of quantum theory too. I think what you mean is what happens *over time*. In classical physics, given the state of the system at time t=0, there is only one possible state of the system at (e.g.) t=1. Quantum theory introduces non-determinism. (This is different from an equation somehow giving different answers on its right-hand side to given values on its left-hand side. In the examples you give - F=ma and E=mc^2 - the two sides of the equation are both talking about the same moment in time.)

Line 7: "An example of the probabilistic nature of quantum mechanics is quantum superposition". Not true - being in a superposition of |0> and |1> is not the same as having different probabilities of being in |0> or |1>. It is only with measurement that probablitiy enters the picture.

Third paragraph: Make it clear that faster-than-light transfer of information is till impossible in quantum theory (the no-signalling theorem), and that quantum teleportation is not teleportation - it is just a colorful name.

Section 4:

"In addition to this will 10 qubits require 2^10 binary combinations to be calculated but also displayed." - 'complex numbers' would be better than 'binary combinations' (which suggests displaying just the sequences 0000...0, 000...01, etc.) A state requires 2^10 complex numbers hence 2^11 floating point numbers, and the matrix of a 10-qubit circuit requres 2^20 complex numbers or 2^21 floating point numbers!

Line 7: "the qubits’ probability" - better "the qubits' state". The state of a quantum system is not the same as probability.
'
Section 5.2

Line 2: "REST-full" -> "REST-ful"

Line 5: "react" -> "React"

Section 5.3

The explanation of how you use the Kronecker product is confusing at the moment. If you want to explain this here, it would be good to use mathematical formulas, rather than try to explain everything using words.

"The program transforms each step to be able to calculate the dot product between the state vector and each step". Unclear. Are you talking about applying a gate to a state vector? To do this we multiply the gate's matrix by the state vector - this is not a dot product.

"We then apply the Kronecker product between each matrix in the step which results in a matrix of size n × n where n is the number of qubits, representing the step." - The matrix will be of size 2^n x 2^n.

Section 6

It's good that you're thinking about the ethical implications of quantum computing. You don't have 

RE the QUESTION TO ROBIN - no, no need to cite a source on whether furthering artificial intelligence will be a good thing. You've already gone into more detail than necessary.

Section 7

Version 2: "the probability vectors" -> "the state vectors". The coordinates of a state vector are not probabilities.

Version 3: "Built-in algorithms in the program" - not clear what this means. I think you mean you want to implement several known quantum algorithms using your system.

Best,

/Robin